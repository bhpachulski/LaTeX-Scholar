\documentclass{myarticle}

\title{}
\author{}

\begin{document}

\maketitle

\begin{abstract}

\vspace{.3cm}
\textbf{Palavras-chave:}
\end{abstract}



\section{Introdu��o}

As seguintes conven��es s�o adotadas neste texto: palavras em caracteres mono-espa�ados s�o comandos
de sistema; palavras em idioma estrangeiro (ingl�s) est�o em it�lico; nomes de programas est�o em
letras normais. Alguns termos em portugu�s ser�o acompanhados de sua tradu��o em ingl�s ou a
nomenclatura utilizada em software livre: nesse caso, o texto estar� em it�lico e entre par�nteses.
No exemplos que cont�m scripts (aqueles que se iniciam com o caracter `\#'), o caracter
`$\backslash$', quando encontrado no final da linha, identifica que a linha continua na linha
abaixo.



\section{Entidades externas}



\section{Conclus�es}


\bibliographystyle{abnt-alf}
\bibliography{root}


\end{document}
