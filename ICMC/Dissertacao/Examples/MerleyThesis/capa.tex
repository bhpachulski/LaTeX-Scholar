\begin{titlepage}


\ \vfill

\begin{center}
\begin{minipage}[c]{12cm}
\begin{center}
\hrulefill\\
\vspace{.5cm} {\Large O efeito do uso de diferentes formas de extra��o de termos na compreensibilidade e representatividade dos termos em cole��es textuais na l�ngua portuguesa}\\
%Gera��o de termos a partir de cole��es textuais utilizando diferentes t�cnicas de simplifica��o de termos
\vspace{1.3cm}
\textbf{\it Merley da Silva Conrado}\\
\vspace{.5cm}
\hrulefill\\
\end{center}
\end{minipage}
\end{center}

\vfill

\cleardoublepage


\begin{flushright}
\begin{Sbox}
\begin{minipage}{8.5cm}
\footnotesize
SERVI�O DE  P�S-GRADUA��O DO ICMC-USP\\
\\
Data de Dep�sito:    29/07/2009\\
\\
Assinatura:\hrulefill
\end{minipage}
\end{Sbox}
\fbox{\TheSbox}
\end{flushright}


\vspace*{3cm}
\begin{center}
{\huge\sf O efeito do uso de diferentes formas de extra��o de termos na compreensibilidade e representatividade dos termos em cole��es textuais na l�ngua portuguesa}


\vspace*{2cm}

{\it Merley da Silva Conrado}

\vspace*{3cm}


{\bf Orientadora:}  {\it Prof� Dr� Solange Oliveira Rezende}

\end{center}

\vspace*{4cm}

\begin{flushright}
\begin{minipage}{10cm}
Disserta��o apresentada ao Instituto de Ci�ncias Matem�ticas e de
Computa��o  - ICMC-USP, como parte dos requisitos para obten��o
do t�tulo de Mestre em Ci�ncias de Computa��o e Matem�tica Computacional.
\end{minipage}
\end{flushright}

\vspace*{3cm}
\begin{center}
\textbf{USP - S�o Carlos \\Julho/2009}
\end{center}
\cleardoublepage



\end{titlepage}
