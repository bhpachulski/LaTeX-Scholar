\chapter{Avalia��o do Processo para Desenvolvimento de Projetos de Pesquisa}

\section{Considera��es Iniciais}

De acordo com Humphrey, a �ltima etapa da defini��o de processos
se refere � valida��o da proposta. O autor prop�e que seja
realizada uma execu��o simulada do processo usando dados de um
projeto que tenha sido completado recentemente. Em seguida, o
processo deve ser utilizado para o desenvolvimento de um pequeno
projeto ou de um prot�tipo.

Conforme apresentado na Se��o \ref{aval1}, para a realiza��o da
execu��o simulada do processo, foi escolhido o projeto SAFE
({\it Software Engineering Available for Everyone}), desenvolvido no
per�odo de 2004 a 2006 por pesquisadores e estudantes de duas
universidades e por profissionais de uma empresa de
desenvolvimento de software. A execu��o do processo no
desenvolvimento de um projeto n�o foi realizada.

Para complementar a valida��o da proposta foi realizada uma
avalia��o da estrutura do processo seguindo os crit�rios
estabelecidos por Horn \cite{hor90} apud Humphrey \cite{Hum95},
que se referem a princ�pios de mapeamento de informa��o. Os
resultados obtidos s�o apresentados na Se��o X. Al�m disso, foi
realizada uma pesquisa com participantes do desenvolvimento de
projetos de pesquisa em que foi perguntado quais as principais
dificuldades que tiveram e como eles julgavam que um processo
poderia colaborar para minimiz�-las. Os resultados s�o
apresentados na Se��o x (emails + sbes).

\section{O processo proposto e os dados do projeto SAFE} \label{aval1}

Aqui preciso ver se vou deixar o texto que esta abaixo ou se vou
usar a versao que foi enviada pro clei, que est� resumida a
revisada. Acho que vai ser preciso intercalar partes das duas. 

\section{Caracter�sticas Gerais do Projeto} \label{carac}

O projeto SAFE teve in�cio em outubro de 2004 e foi finalizado em
dezembro de 2006. Seu principal objetivo foi desenvolver uma
infra-estrutura para permitir a integra��o de ferramentas de
software livre de apoio �s atividades de Engenharia de Software,
possibilitando um suporte automatizado para o processo de software
livre. O projeto foi financiado pela FINEP.

%Inicialmente o projeto deveria ser desenvolvido no per�odo de um ano, por�m, ocorreram atrasos na durante a fase de aquisi��o dos equipamentos necess�rios para o desenvolvimento e assim o prazo foi extendido (os atrasos ocorreram principalmente porque ajustes foram necess�rios no documento de aquisi��o de equipamentos (como chama o documento??) de forma a cumprir integralmente a lei vigente no instituto. Desconsiderando-se, portanto, o prazo necess�rio para a aquisi��o dos equipamentos solicitados pode-se afirmar que o projeto cumpriu o prazo estabelecido para seu desenvolvimento.

No contexto acad�mico, participaram do projeto professores e
estudantes do ICMC-USP (institui��o executora) e da UFMS
(institui��o co-executora). No setor empresarial, houve
participa��o de membros da empresa ASYNC (interveniente), situada
em S�o Carlos, cuja principal atividade � o desenvolvimento de
sistemas de software utilizando plataformas de software livre. O
n�mero de partipantes foi bastante vari�vel durante todo o
desenvolvimento do projeto sendo que, em m�dia, aproximadamente 15
pessoas participaram.

Caracter�sticas importantes do desenvolvimento do projeto SAFE se
referem ao cumprimento de prazos e custos. Considerando-se o
per�odo a partir do qual o projeto come�ou a ser efetivamente
desenvolvido, pode-se considerar que o prazo estabelecido para seu
desenvolvimento foi cumprido. Deve-se ressaltar que o in�cio do
desenvolvimento do projeto teve um atraso de aproximadamente X
meses. Isto � justificado pelo tempo gasto na fase de aquisi��o
dos equipamentos (diversos ajustes foram necess�rios no documento
relativo ao pedido de compra, de forma a atender integralmente as
leis vigentes nas institui��es associadas). Esses meses
necess�rios para a aquisi��o dos equipamentos n�o havia sido
inclu�do no cronograma, o que gerou problemas para o cumprimento
das metas de acordo com os prazos inicialmente estabelecidos. Como
consequ�ncia, foi solicitado e concedido a extens�o do prazo para
o desenvolvimento do projeto. Considerando-se, portanto, a
concess�o que ocorreu, os prazos foram definidos novamente e foram
cumpridos pelas equipes.  � importante ressaltar tamb�m que o
or�amento foi cumprido conforme previsto. Os dados relativos a
prazos e custos do projeto foram apresentados no primeiro
relat�rio t�cnico do projeto SAFE, submetido � ag�ncia
financiadora em 12/12/2005 e aprovado em e no relat�rio final do
projeto, submetido � ag�ncia financiadora em 15/12/2006, ainda em
avalia��o.

� interessante analisar tamb�m os resultados obtidos a partir do
desenvolvimento do SAFE. No desenvolvimento de projetos de
pesquisa a qualidade cient�fica dos resultados � um fator
primordial. Dessa forma, ao inv�s de analisar atributos de
qualidade de produto de software, como aqueles indicados na Norma
ISO/IEC 9126 \ref{ISO9126a}, optou-se por utilizar como indicador
de qualidade as publica��es obtidas no contexto do projeto. Assim,
como resultado do projeto SAFE foram obtidas 12 publica��es em
confer�nicas nacionais, 1 publica��o em revista nacional e 5
publica��es em confer�ncias internacionais; foram gerados 8
relat�rios t�cnicos e foram desenvolvidos 13 projetos de inicia��o
cient�fica e 5 projetos de mestrado e 3 projetos de doutorado.

Nas pr�ximas sub-se��es s�o apresentadas descri��es breves das
fases cumpridas para o desenvolvimento do projeto SAFE e �
apresentado o relacionamento com o processo proposto.

\subsection{A Concep��o do Projeto e a Fase de Aquisi��o}

O tema de software livre, que � o tema principal do projeto SAFE,
come�ou a ser estudado a partir do desenvolvimento do trabalho de
mestrado de Reis \cite{Reis03}. Os resultados obtidos naquela
ocasi�o motivaram a continuidade das investiga��es por uma equipe
maior, em que pessoas possuindo diferentes compet�ncias pudessem
colaborar no intuito de aprofundar os conhecimentos e gerar novas
solu��es para os problemas identificados no trabalho mencionado.
Assim, percebendo-se a motiva��o de um grupo de pessoas do
ICMC-USP, da UFMS e da empresa ASYNC em rela��o ao tema, foi
elaborado um projeto de pesquisa que foi inicialmente submetido ao
CNPq na chamada X. O pedido n�o foi contemplado, no entanto,
alguns dias ap�s a divulga��o do resultado, foi aberto edital para
submiss�o de propostas de projetos de pesquisa para financiamento
pela FINEP, sendo que o tema dos projetos deveria estar
relacionado a software livre. Assim, foram realizadas altera��es
na proposta de projeto inicial e uma nova submiss�o foi realizada.
Cento e vinte e tr�s (correto?) projetos foram submetidos, dos
quais apenas 16 foram selecionados. O projeto SAFE foi um dos
contemplados com o financiamento pela FINEP nesta chamada.

Durante a fase de elabora��o do projeto foram preenchidos diversos
formul�rios que foram solicitados pela pr�pria ag�ncia. Foram
apresentados o objetivo geral do projeto, os objetivos espec�ficos
e metas f�sicas que pudessem indicar o cumprimento desses
objetivos, justificativa resumida e detalhada do projeto,
metodologia para desenvolvimento da proposta, contendo a descri��o
das tarefas, resultados esperados, identifica��o de metas que
deveriam ser atingidas, planejamento da ger�ncia do projeto,
mecanismos de transfer�ncia de resultados e divulga��o do projeto,
impacto econ�mico, cient�fico, tecnol�gico e social, identifica��o
das pessoas envolvidas com o desenvolvimento do projeto,
elabora��o do cronograma f�sico (relacionamento entre metas,
atividades para cumprimento de cada meta, indicador f�sico
relacionado, data de in�cio e de t�rmino de cada atividade) e
descri��o de recursos financeiros solicitados. Neste ponto
observa-se uma diferen�a expressiva entre projetos desenvolvidos
com fins comerciais e projetos de pesquisa. Como observado por
Bellotti et al \cite{bel02}, em ambiente comercial o valor
estrat�gico de um projeto geralmente � facilmente compreendido, de
forma t�cita, por todos os participantes, pois h� um cliente que
encomenda e paga por um sistema que atenda �s necessidades de seus
neg�cios. Ao contr�rio, em ambiente de pesquisa, durante a fase de
elabora��o de um projeto o objetivo � identificar e apresentar, de
forma clara, o valor estrat�gico do projeto. De fato, na
experi�ncia do projeto SAFE, as informa��es solicitadas indicam a
import�ncia em explicitar o valor cient�fico do projeto, os
resultados pr�ticos provenientes de seu desenvolvimento e o
impacto causado considerando-se diferentes perspectivas.

Ap�s a aprova��o do projeto e assinatura de contrato pelos
respons�veis (coordenadores do projeto nas duas universidades
envolvidas, membro do setor empres�rio, diretor das universidades
nas quais foi desenvolvido, diretor da ag�ncia de financiamento --
foram estes mesmo? mais algu�m? ) iniciou-se o processo de
aquisi��o de equipamentos, o qual se prolongou por X meses. As
fases que constaram deste processo foram: defini��o da quantidade
e da configura��o completa dos equipamentos que precisavam ser
adquiridos, encaminhamento de pedido ao setor de compras,
aprova��o do pedido por membros da reitoria das institui��es
envolvidas, abertura de licita��o de acordo com a lei vigente,
aquisi��o dos equipamentos, entrega, instala��o e patrimoniamento
dos equipamentos. De acordo com Enza (? -- preciso fazer uma
entrevista com ela!), estes processos s�o gen�ricos e s�o
cumpridos para a aquisi��o de bens dur�veis(correto?) no contexto
de todos os projetos de pesquisa desenvolvidos no instituto. Em
rela��o � aquisi��o de materiais de consumo e outros tipos de
materiais, por exemplo, livros, normas t�cnicas e software cujos
valores n�o ultrapassem limites estipulados pela institui��o, �
realizada cota��o com tr�s fornecedores e, em seguida, o material
� adquirido por interm�dio do fornecedor que apresentar menor
pre�o para a solicita��o. Durante a elabora��o do projeto n�o foi
considerada a atividade referente a fase de aquisi��o dos
equipamentos e de outros materiais fundamentais para o
desenvolvimento do projeto e, portanto, n�o foi alocado tempo para
o cumprimento desta fase. No entanto, conforme apresentado na
Se��o \ref{carac}, a execu��o das tarefas de aquisi��o se
prologaram e causaram atrasos no cronograma proposto. Assim,
sugere-se alocar recursos (em termos de tempo) para a fase de
aquisi��o e tomar conhecimento pr�vio das leis que regulam a
aquisi��o de materiais nas institui��es envolvidas com o projeto
em desenvolvimento.


\subsection{Ciclo de Vida} \label{cic}

No in�cio do projeto foram realizadas diversas reuni�es para que os membros envolvidos pudessem compreender o objetivo do projeto e come�ar a identificar os requisitos iniciais. Os participantes foram motivados a expressar suas opini�es a respeito da proposta aprovada, identificar casos de uso do sistema e indicar poss�veis direcionamentos para o desenvolvimento do projeto (uso de metodologias, ferramentas, plataformas e arquiteturas). A maioria das reuni�es foi realizada em formato de {\it brainstorming}, cuidando-se do planejamento e do registro (por meio de atas) dos �tens apresentados e discutidos em cada reuni�o. Emails eram enviados a uma lista criada para o projeto contendo informa��es relativas ao planejamento e ao registro das informa��es das reuni�es. Como resultado dessas reuni�es foram gerados diagramas de casos de uso, descri��o dos casos de uso e modelo de dados pelos membros do ICMC-USP e da empresa associada.

Em seguida foi elaborado um projeto simples do sistema e um prot�tipo foi desenvolvido pelos membros da UFMS. Os resultados obtidos foram discutidos pelos partipantes no I Workshop do projeto SAFE, realizado em X/X/2005 em no campus da USP/S�o Carlos em que estiveram presentes representantes das duas universidades envolvidas e do setor empresarial. Foram tomadas diversas decis�es relacionadas � continuidade do projeto, que tamb�m foram registradas em formato de ata e enviadas por email aos participantes.

Com o desenvolvimento do prot�tipo foi poss�vel entender melhor e refinar os requisitos do sistema. Como consequ�ncia, atualiza��es foram realizadas nos diagramas inicialmente gerados. Com esta evolu��o, decidiu-se por dividir o trabalho de implementa��o entre os membros de cada entidade envolvida, de forma que os trabalhos pudessem ser desenvolvidos de forma colaborativa. Os resultados obtidos foram apresentados no II workshop do projeto SAFE realizado em 16, 17 e 18 de novembro de 2005 no campus da UFMS. Novamente estiveram presentes membros das tr�s entidades envolvidas.
Com as discuss�es ocorridas no II workshop, os casos de uso do sistema foram novamente refinados e a implementa��o do sistema continuou a evoluir.

As reuni�es continuaram sendo realizadas no decorrer do desenvolvimento. Pelo fato de a equipe da UFMS ser menor, as reuni�es eram mais informais, geralmente entre o coordenador do projeto na institui��o e um aluno envolvido. No ICMC foi mantido o planejamento e a realiza��o das reuni�es. Atas continuaram a ser elaboradas, de forma a divulgar entre todos os membros o que havia sido discutido e as decis�es tomadas (por exemplo, posso anexar o email da Renata de 31/01/2005, contendo o planejamento da reuniao e a ata do Luciano).

Alguns autores indicaram ser fundamental a ado��o de um ciclo de vida iterativo para o desenvolvimento de projetos de pesquisa \cite{seg05,Oli05}, principalmente devido ao fato de que os cientistas s�o experientes em desenvolver software de forma bastante iterativa, percebendo os requisitos emergir em sucessivas itera��es. Para o desenvolvimento do projeto SAFE n�o foi estabelecido inicialmente um modelo de ciclo de vida a ser utilizado e, na pr�tica, a abordagem iterativa sugerida na literatura foi adotada. A experi�ncia foi interessante principalmente em rela��o � ``descoberta'' dos requisitos do sistema. No entanto, durante a realiza��o dos workshops, em que as fases do ciclo de vida cumpridas foram apresentadas, sentiu-se a necessidade da elabora��o de outros diagramas (principalmente diagrama de classes e arquitetura do sistema), de forma a facilitar o entendimento e a evolu��o do projeto. � importante considerar que o desenvolvimento foi realizado de forma distribu�da e colaborativa e, portanto, membros de diferentes institui��es comumente precisavam entender e dar continuidade a partes do sistema desenvolvido por outros membros. Outro problema identificado foi a falta de documenta��o de testes realizados, que gerava incertezas a respeito de o quanto a implementa��o havia sido avaliada. Este problema foi considerado um dos principais do projeto, principalmente pelo fato de o projeto ter sido desenvolvido em um ambiente colaborativo.

Considerando-se as recomenda��es da literatura e a experi�ncia no projeto SAFE, sugere-se a ado��o de um ciclo de vida iterativo, que considere as fases de defini��o e desenvolvimento do sistema. Uma alternativa interessante e que tem sido adotada para o desenvolvimento de projetos de pesquisa � o uso de metodologias �geis (refs).


\subsection{Transfer�ncia de conhecimento entre os participantes}

Uma atividade bastante interessante no contexto do projeto foi a transfer�ncia de conhecimento entre as pessoas envolvidas, que ocorreu, principalmente, devido a dois fatores:

\begin{enumerate}

\item[(1)] em ambiente acad�mico � comum a alta rotatividade dos membros de um projeto e isso de fato ocorreu durante o desenvolvimento do projeto SAFE;

\item[(2)] durante a etapa de aquisi��o de equipamentos, que se prolongou por x meses, membros de uma das institui��es come�aram a pesquisar a potencialidade de algumas ferramentas que poderiam ser utilizadas no contexto do projeto. Quando o projeto come�ou a ser desenvolvido de fato, estes membros j� possu�am familiaridade com diversos elementos relacionados ao tema do projeto e optou-se pela transfer�ncia do conhecimento adquirido para os demais membros.

\end{enumerate}

As atividades cumpridas com o objetivo de viabilizar a transfer�ncia de conhecimento entre os participantes estiveram relacionadas basicamente � elabora��o de documenta��o sobre os t�picos estudados no formato de relat�rio t�cnico e a realiza��o de cursos para os membros. Por exemplo, isso ocorreu com a ferramenta Bugzilla, que come�ou a ser estudada inicialmente por alguns membros da equipe. Gerou-se um relat�rio t�cnico contendo os objetivos e os recursos da ferramenta, suas principais caracter�sticas e vis�o geral de suas funcionalidades \cite{Quem05}. O relat�rio foi disponibilizado aos membros e um minicurso foi realizado no I workshop. Observa-se ainda que o material gerado foi utilizado em diversas outras ocasi�es, por exemplo, em disciplinas do curso de Ci�ncia da Computa��o das duas institui��es \cite{wei}, em outros projetos de pesquisa \footnote{O projeto Tidia-AE e o projeto do Castelo utilizam os recursos disponibilizados pelo projeto SAFE, que inclui documenta��o, material para apresenta��o de cursos e reposit�rios} e em cursos de extens�o ministrados para a comunidade, o que possibilitou tamb�m a transfer�ncia do conhecimento adquirido para outras entidades externas ao projeto.

\subsection{Planejamento e Gerenciamento do Projeto}

No in�cio do desenvolvimento do projeto algumas planilhas foram geradas com o objetivo de servirem como base para o planejamento inicial e o gerenciamento das atividades do projeto. Desse modo, utilizando-se o sistema OpenOffice, foram geradas 3 planilhas contendo o relacionamento entre as metas do projeto e as pessoas respons�veis; o relacionamento entre metas e prazos para cumprimento e o relacionamento entre metas, prazos para cumprimento e artefatos que deveriam ser elaborados. As atividades cumpridas pelos participantes para o cumprimento das metas eram discutidas em reuni�es do projeto. Eventualmente, era solicitado (por email enviado � lista) que os participantes indicassem as atividades que estavam cumprindo e os resultados que estavam sendo obtidos (por exemplo, posso anexar o email do Luciano de 31/01/2005 - emails estao no gmail tambem).

As planilhas foram �teis na atribui��o inicial das metas aos participantes e na defini��o de prazos e de artefatos a serem gerados (de forma mais detalhada que na fase de elabora��o do projeto). Observando-se que a atualiza��o das planilhas estava se tornando uma tarefa pouco eficiente a medida que o volume de informa��es aumentava, optou-se por utilizar um sistema de gerenciamento online, no caso, o sistema NetOffice. As pessoas envolvidas foram motivadas a acrescentar informa��es referentes ao desenvolvimento de suas atividades, de forma a evitar que um dos membros precisasse solicitar as informa��es por email e fazer as atualiza��es, conforme estava ocorrendo anteriormente. Apesar de a equipe ter sido treinada para usar a ferramenta e a import�ncia em utiliz�-la ter sido enfatizada, poucas informa��es de fato foram registradas. Grande parte das informa��es continuou a ser registrada por um dos membros do projeto, de acordo com as atividades e os resultados apresentados nas reuni�es. De forma similar ao que acontece em ambiente industrial, notou-se que n�o houve entre os participantes a cultura de documentar, passo a passo, as atividades que cumprem. No geral, apenas os resultados finais s�o apresentados, geralmente na forma de um artefato.

Apesar de o registro das informa��es no sistema de gerenciamento n�o ter ocorrido de forma ideal, ainda assim considerou-se bastante proveitosa sua utiliza��o. O cumprimento de atividades alocadas a cada participante p�de ser acompanhada mais facilmente pois outras informa��es como prioridade da atividade no contexto do projeto (nenhuma, muito baixa, baixa, media, alta, m�xima), estado (aberto, finalizado pelo cliente, finalizado, n�o iniciado, suspenso), taxa de finaliza��o da atividade, (em porcentagem), data de in�cio e de t�rmino e tempo estimado para cumprimento de cada atividade estavam dispon�veis. Outros tipos de informa��es tamb�m podiam ser acrescentadas, como discuss�es da equipe, anota��es e relat�rios. Outra funcionalidade interessante da ferramenta e que foi bastante �til no contexto do projeto foi a gera��o de gr�ficos apresentando a rela��o entre atividades e prazos estabelecidos.
Anexo X (posso colocar um grafico gerado atividades X prazos).

Em rela��o ao gerenciamento de custos do projeto, adotou-se a metodologia de registrar em planilha, durante todo o desenvolvimento do projeto, todos os pagamentos efetuados. Como apresentado no Anexo, esta planilha era composta por descri��o da despesa e valor pago (mais algum??). O valor total gasto e o valor dispon�vel era calculado a cada pagamento efetuado. Todos os extratos banc�rios e notas fiscais foram arquivados para comprovar, junto � ag�ncia financiadora, os investimentos realizados. Considerou-se suficiente a gera��o da planilha dado que o projeto SAFE possu�a pequeno porte.

\subsection{Divulga��o do projeto} \label{div}

Conforme apresentado na Se��o \ref{cola}, um mecanismo importante utilizado para promover a divulga��o do projeto foi a disponibiliza��o de informa��es na internet, utilizando-se o sistema coweb e posteriormente o portal do projeto (http://safe.icmc.usp.br). Isso possibilitou que o projeto fosse divulgado em �mbito nacional e, como resultado, pessoas de diferentes institui��es entraram em contato com membros da equipe buscando conhecer mais o projeto e estabelecer parcerias para o desenvolvimento de novos projetos relacionados ao mesmo tema do projeto SAFE (posso colocar o exemplo de email da unifacs querendo submeter uma proposta).

Outra forma de divulgar o projeto foi a realiza��o de cursos de extens�o para estudantes de gradua��o e de p�s gradua��o das institui��es envolvidas e para a comunidade externa. Foram oferecidos cursos sobre ferramentas de software livre e foram apresentadas palestras sobre o pr�prio projeto em disciplinas de apresenta��o de semin�rios nas duas institui��es. Al�m disso, o projeto foi apresentado em eventos cient�ficos, na forma de apresenta��o de artigos e de exposi��o do projeto em outros tipos de reuni�es cient�ficas, por exemplo, em mesa redonda em simp�sio. %e participa��o  em outras reuni�es, que envolveram, por exemplo, a apresenta��o do projeto em simp�sio promovido pela SBC e em reuni�es promovidas pela FINEP.

\subsection{Documenta��o do Projeto}

Al�m dos artefatos gerados como resultado do ciclo de vida do projeto (Se��o \ref{cic}) optou-se pela elabora��o de uma cartilha, contendo a descri��o da pesquisa realizada no contexto do projeto. Na cartilha n�o foi enfatizado o sistema desenvolvido e sim as etapas cumpridas para a realiza��o das pesquisas subjacentes ao projeto. Considerou-se a elabora��o deste documento importante por tornar expl�cito o embasamento cient�fico do projeto
e outras quest�es relacionadas ao tema, por exemplo, como as fases de um projeto de software livre s�o geralmente cumpridas (estas informa��es s�o fundamentais para que os usu�rios possam entender o projeto e utilizar o sistema desenvolvido).

Dois relat�rios t�cnicos tamb�m foram gerados com o objetivo principal de viabilizar a presta��o de contas � ag�ncia financiadora. Estes relat�rios tamb�m servem como documenta��o do projeto, a medida em que descrevem as atividades cumpridas e os resultados obtidos, sob diferentes perspectivas (p. exemplo??).

\subsection{Suporte e infra-estrutura}

A medida que o projeto evoluiu e as ferramentas passaram a ser utilizadas por outros usu�rios, sentiu-se a necessidade de adotar uma pol�tica de backup. A pol�tica adotada foi a seguinte:

\subsection{Revis�o bibliogr�fica e reda��o de artigos}

Durante o desenvolvimento de todo o projeto os participantes foram motivados a ler artigos e documentos cient�ficos relacionados ao tema. Isso foi muito importante porque, al�m de auxiliar no entendimento do estado da arte em termos de pesquisas realizadas sobre o tema, indicou um conjunto de ferramentas que n�o eram conhecidas pelos membros e que foram �teis para o projeto. Era comum, durante a realiza��o das reuni�es do projeto, que artigos fossem apresentados e discutidos pela equipe.

Os membros foram estimulados tamb�m a redigir e submeter artigos apresentando os resultados obtidos. Conforme observado na Se��o \ref{div}, este foi um mecanismo importante utilizado para a divulga��o do projeto e, ao mesmo tempo, uma forma de apresentar � comunidade os resultados dos esfor�os empregados e dos investimentos realizados.

\subsection{Caracter�sticas de um Processo Distribu�do} \label{dist}

Para identificar o quanto o projeto de fato cumpriu um processo distruibu�do e tornar expl�cita a experi�ncia dos grupos em rela��o � processos distribu�dos, foi observado se elementos b�sicos de um processo desse tipo, identificados por Maidantchik \cite{Mai99} como resultado da an�lise de alguns trabalhos apresentados na literatura, foram considerados no contexto do projeto SAFE.  Os itens destacados em it�lico a seguir se referem aos requisitos que a autora indica como fundamentais para que um processo distribu�do seja cumprido efetivamente.

\begin{enumerate}

\item {\it Determinar a capacidade das equipes:} durante a fase de elabora��o do projeto foram coletadas e registradas informa��es sobre forma��o e �rea de atua��o de cada membro. As seguintes informa��es foram apresentadas no primeiro formul�rio submetido � ag�ncia financiadora durante a submiss�o da proposta do projeto: titula��o, institui��o associada, �rea de especializa��o, fun��o no projeto, participa��o no projeto (quantidade de horas de trabalho por semana, quantidade de meses, custeio) e atividades para as quais o membro foi alocado. Os recursos tecnol�gicos e econ�micos de cada equipe tamb�m foram indicados, descrevendo-se a infra-estrutura f�sica de cada uma das institui��es e da empresa associada, financiamentos obtidos com o desenvolvimento de outros projetos, atividades de pesquisa e desenvolvimento realizadas, produ��o cient�fica e tecnol�gica e atividades de extens�o realizadas. O cadastro dos  grupos foi atualizado no primeiro relat�rio t�cnico do projeto.

\item {\it Apoiar a coordena��o distribu�da:} cada equipe do ICMC-USP foi supervisionada localmente por um coordenador que era um aluno de p�s gradua��o. A coordena��o local n�o foi atribu�da a apenas uma pessoa, ou seja, diferentes membros atuaram como coordenador durante o desenvolvimento do projeto. A defini��o das responsabilidades de cada participante foi realizada em conjunto com a coordenadora do projeto. Para acompanhar a realiza��o das tarefas foram realizadas reuni�es periodicamente e as mudan�as no projeto, decis�es e problemas discutidos foram registrados em atas, conforme apresentado na Se��o \ref{cic}. Na UFMS e na empresa ASYNC, por haver um n�mero menor de partipantes, n�o foi atribu�da a tarefa de coordena��o a algum dos membros.

\item {\it Apoiar a ger�ncia distribu�da do projeto:} O uso da ferramenta NetOffice apoiou a ger�ncia distribu�da do projeto. Apesar do uso restrito pelos participantes, as informa��es eram atualizadas a medida que os resultados eram apresentados em reuni�es, quando as atividades eram explicitamente atribu�das aos participantes ou quando os artefatos eram finalizados. A supervis�o das atividades cumpridas foi realizada principalmente por meio da revis�o dos artefatos gerados durante o desenvolvimento do projeto. Pontos de controle do projeto n�o foram explicitamente definidos, por�m, a realiza��o dos workshops e os {\it deadlines} para submiss�o de relat�rios t�cnicos serviram como importantes marcos do projeto, em que diversos pontos foram avaliados e decis�es relacionadas � continuidade do projeto foram tomadas.

\item {\it Apoiar o controle de artefatos:} n�o foram avaliados os procedimentos que estavam sendo utilizados para a produ��o dos artefatos. A qualidade dos artefatos era avaliada por meio de revis�es, que ocorriam quando resultados intermedi�rios eram obtidos ou quando os artefatos eram finalizados. O controle de vers�es dos artefatos produzidos pelas equipes foi realizado utilizando-se uma das ferramentas em estudo no projeto, a ferramenta Subversion. N�o foi definido um processo detalhado para controle dos artefatos envolvendo planejamento, monitora��o e notifica��o de resultados, muda�as, inconsist�ncias e depend�ncias. Foi adotada apenas uma pol�tica de permiss�o de acesso aos servidores e �s bases de dados das ferramentas.

\item {\it Apoiar a comunica��o entre as equipes:} os principais recursos utilizados para promover a comunica��o entre os membros foram emails, lista de discuss�o e mensagens instant�neas. Durante todo o desenvolvimento do projeto, principalmente durante o cumprimento das atividades do ciclo de vida, houve a possibilidade de utilizar os recursos de comunica��o da ferramenta Bugzilla, no entanto, na pr�tica estes recursos foram pouco explorados pelos membros.

\item {\it Apoiar a publica��o e o compartilhamento de informa��es:} a publica��o de informa��es para a comunidade externa foi apresentada na Se��o \ref{div}. Entre os membros da equipe foram utilizados tamb�m os recursos ``Events'', ``News'' e ``Meetings'' do portal para divulgar reuni�es, atas, mudan�as no projeto e resultados alcan�ados (o sistema coweb tamb�m foi utilizado neste sentido antes da disponibiliza��o do portal).

\item {\it Apoiar a resolu��o de problemas:} n�o foram incorporadas ao processo atividades para notificar a exist�ncia de um problema, divulgar a sua resolu��o e identificar as poss�veis implica��es no trabalho das equipes. As notifica��es eram realizadas informalmente, durante a realiza��o de reuni�es, ou a partir de resultados das discuss�es realizadas nos workshops.

\item {\it Incorporar um reposit�rio de terminologias:} n�o foi documentado um gloss�rio sobre o dom�nio da aplica��o do projeto, de forma a possibilitar a compreens�o �nica dos termos pelas equipes envolvidas. Ao inv�s disso, no in�cio do projeto foram realizadas diversas reuni�es entre os membros do ICMC-USP e da empresa ASYNC (que s�o os membros mais experientes em desenvolvimento de software livre) com o objetivo de explicar os principais termos da �rea para a equipe. Essas reuni�es foram muito importantes em termos t�cnicos e, al�m disso, serviram para motivar a equipe em torno do tema. � importante mencionar tamb�m que, apesar de n�o haver um gloss�rio que fosse atualizado constantemente, o documento gerado por Reis \cite{Reis03} foi utilizado como uma fonte bibliogr�fica importante em que a defini��o de muitos termos foi apresentada.

\item {\it Apoiar a redefini��o de atividades de software:} conforme apresentado na Se��o \ref{cic}, as atividades do processo foram cumpridas de forma iterativa, no entanto, n�o foi documentado formalmente um processo que devesse ser seguido por todos os membros (ou adaptado de acordo com as caracter�sticas e restri��es das equipes). Quanto � infra-estrutura necess�ria para a realiza��o de tarefas de forma distribu�da foram disponibilizados sistema de ger�ncia de vers�es (Subversion), controle de altera��es (Bugzilla), gerenciamento (NetOffice) e documenta��o (DocRationale).


\end{enumerate}

\subsection{Caracter�sticas de um Processo Colaborativo} \label{cola}

Augustin et al \cite{Aug02}, Pinheiro et al \cite{Pin02}, Arnold \cite{Arn95} e Cook et al \cite{coo04} indicaram v�rios fatores que s�o importantes no contexto de desenvolvimento de projetos colaborativos, que s�o: defini��o de pap�is, adapta��o dos diferentes processos cumpridos pelas equipes para um ambiente de colabora��o, representa��o do processo e dos pontos de colabora��o que devem ser compartilhados entre os membros do grupo, integra��o de c�digo e de outros artefatos, ger�ncia de configura��o, uso de ferramentas de edi��o colaborativa, uso de ferramentas de comunica��o s�ncronas e ass�ncronas, coordena��o das atividades, registro do conhecimento comum ao grupo e percep��o do grupo em rela��o ao contexto de trabalho. Estes fatores ser�o analisados a seguir em rela��o ao desenvolvimento do projeto SAFE.

A defini��o de pap�is ocorreu durante a fase de elabora��o do projeto, em um sentido mais amplo, ou seja, foram identificados os coordenadores locais, os pesquisadores (mais algum?). No entanto, n�o foram identificados pap�is para o desenvolvimento das atividades t�cnicas do projeto (por exemplo, analistas, programadores, testadores, etc). As pessoas envolvidas tinham a liberdade de assumir os pap�is que julgassem mais convenientes de acordo com suas experi�ncias e aspira��es. Um problema observado foi que, em alguns casos, n�o houve comprometimento dos membros em assumir os pap�is que eles pr�prios haviam escolhido.

Em rela��o ao processo cumprido, n�o foi realizado inicialmente o entendimento dos processos cumpridos localmente pelas equipes e a adapta��o para um contexto de colabora��o. Deve-se notar tamb�m que n�o houve a representa��o de um processo para o desenvolvimento do projeto e a identifica��o de pontos de colabora��o, ou seja, o processo foi sendo identificado a medida que o projeto evolu�a. % e, como as atividades foram cumpridas de forma iterativa, percebeu-se que

A integra��o entre os artefatos produzidos n�o ocorreu com a frequ�ncia que geralmente � mencionada na literatura (por exemplo, Cook et al \cite{coo04} sugerem {\it a regular global synchronisation -- the nightly build}). As integra��es ocorriam quando partes do projeto eram finalizadas.

A ger�ncia de configura��o n�o foi completamente cumprida, apenas uma de suas atividades, o controle de vers�es, foi priorizada, conforme citado na Se��o \ref{dist}.

Para possibilitar a edi��o colaborativa, optou-se por utilizar inicialmente o sistema coweb cite para o registro de informa��es do projeto (end.). O sistema coweb (falar um pouco). Foram inclu�das neste ambiente informa��es sobre os participantes do projeto (nome, institui��o associada, email), atas das reuni�es e listagem de eventos cient�ficos relacionados ao tema. Alguns participantes inclu�ram tamb�m suas listas de atividades cumpridas e atividades pendentes ({\it ToDo list}), facilitando o gerenciamento de suas pr�prias atividades. Foi apresentado tamb�m o ``Kit de desenvolvimento'', contendo descri��o de metodologias e ferramentas utilizadas no projeto e outras informa��es como site da metodologia ou ferramenta utilizada, dicas, endere�o para download, etc. Portanto, foram inclu�das na coweb informa��es b�sicas relacionadas ao projeto, as quais deveriam ser conhecidas por toda a equipe. Todos os membros estiveram livres para acrescentar outras informa��es que julgassem necess�rias.

Com a evolu��o do projeto, optou-se por disponibilizar um portal utilizando-se o ambiente plone. A altera��o ocorreu porque buscou-se divulgar o projeto de forma mais efetiva para a comunidade externa. O sistema coweb � um sistema educacional que tem sido utilizado por alunos e professores como um sistema que promove a colabora��o durante o desenvolvimento de trabalhos acad�micos e serve como reposit�rio para artefatos educacionais, por exemplo, material did�tico e notas de trabalhos e provas. Portanto, o ambiente plone foi adotado por estar sendo amplamente utilizado em diversos setores, n�o apenas educacionais e,  assim, ser um ambiente mais familiar � comunidade. Outra motiva��o foi que o sistema oferece recursos de colabora��o semelhantes aqueles oferecidos pelo sistema coweb, com o qual os membros do projeto j� estavam acostumados. Deve-se notar que ferramentas para edi��o colaborativa de modelos e de c�digo n�o foram utilizadas.

Conforme mencionado na Se��o \ref{dist} foram utilizadas ferramentas de comunica��o s�ncronas e ass�ncronas e um sistema de gerenciamento de projetos foi adotado. As atividades referentes ao registro do conhecimento comum ao grupo, que trata do registro de todo o processo de intera��o do grupo, foram apresentadas na Se��o x.

Um ponto muito importante em rela��o ao desenvolvimento de sistemas colaborativos � a percep��o do grupo em rela��o ao contexto de trabalho. A percep��o envolve saber quem � o grupo, qual � seu objetivo e quais s�o suas atividades. Envolve o conhecimento sobre o que aconteceu, o que vem acontecendo em rela��o �s atividades do grupo, quem s�o os membros do grupo, onde est�o e o que est�o fazendo. \cite{Pin02} (� esta referencia mesmo??). Em rela��o ao projeto SAFE, o aspecto da percep��o p�de ser explorado com a utiliza��o do portal e do sistema de gerenciamento NetOffice. No portal, muitas informa��es de contexto s�o disponibilizadas. Utilizando-se o ambiente Plone, cada participante do projeto possui sua �rea de trabalho e, portanto, as tarefas cumpridas e os resultados obtidos foram constantemente registrados e disponibilizados aos participantes do projeto. Al�m disso, em algumas ocasi�es (por exemplo, quando algum evento � registrado), o pr�prio sistema solicita e torna expl�cita as informa��es relacionadas a ``what'', ``when'',  ``where'' e ``history''.  (Um exemplo de tela do plone pode ser mostrada). O sistema NetOffice foi importante em rela��o � percep��o por indicar as atividades que estavam sendo cumpridas (relacionadas aos prazos estabelecidos) por cada um dos membros do projeto.

\subsection {Os problemas e as dificuldades encontradas}

- problemas de comunica��o
-



andre me falou sobre o uso do subversion, bugzila foi usada para
discutir pontos de integra��o e IRC.


1- projeto SAFE\\
2- O que os usu�rios esperam de um processo (depoimentos de alunos, professores). \\
depoimentos de alunos -- exemplo: email do tchelo: dificuldades
que os alunos tiveram + depoimentos no SBES\\
3- Avalia��o da estrutura: ver os criterios de Horn para avalia��o
de processos - pg 445, capitulo do humprhey\\
4- metricas definidas na se�ao de objetivos no cap. de concep��o,
comparando com outros processos -- no final do cap. do humprhey
tem um exemplo, no artigo de metricas tambem...
