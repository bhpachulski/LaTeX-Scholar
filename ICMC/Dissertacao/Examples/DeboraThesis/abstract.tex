The increasing volume of research projects in the context of software is a motivation for analyzing processes which are used to develop such projects, regarding activities that are carried out, results that are gathered and resources application. In the Software Engineering area, recently, processes have been considered with the objective to contribute for research projects development in which software is an element. Overall, the main purpose is to help management overtime of artifacts that can be produced, such as, models, code, technical reports and scientific papers. 
However, research developed in this direction and solutions discussed in the literature are presented in an initial stage. Therefore, the objective of this work was to define a process for development of research projects regarding a set of requirements. It is expected that such process can be useful to contribute for evolution of these projects. To reinforce the aspect of research project evolution, the design rationale approach was studied. The goal was to provide the opportunity of capturing and registering decisions in specific development phases. The documentation process was emphasized, i.e., the design rationale approach was analyzed focusing on the improvement of research projects documentation. As a result, a model for design rationale representation was defined, implemented in a CASE tool and evaluated by means of an experiment.
    



 