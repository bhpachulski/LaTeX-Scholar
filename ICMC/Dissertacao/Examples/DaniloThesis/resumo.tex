%%%%%%%%%%%%%%%%%% Resumo %%%%%%%%%%%%%%%%

{\huge{\textbf{Resumo}}}

\hspace{-0.5cm}\rule{\linewidth}{2pt}

\vspace{3cm}

Este trabalho tem como objetivo desenvolver um m�todo de segmenta��o de cenas em v�deos digitais que trate segmentos sem�nticamente complexos. Como prova de conceito, � apresentada uma abordagem multimodal que utiliza uma defini��o mais geral para cenas em telejornais, abrangendo tanto cenas onde �ncoras aparecem quanto cenas onde nenhum �ncora aparece. Desse modo, os resultados obtidos da t�cnica multimodal foram significativamente melhores quando comparados com os resultados obtidos das t�cnicas monomodais aplicadas em separado. Os testes foram executados em quatro grupos de telejornais brasileiros obtidos de duas emissoras de TV diferentes, cada qual contendo cinco edi��es, totalizando vinte telejornais.\\

\textbf{Palavras-chave:} segmenta��o de cenas, segmenta��o sem�ntica, identifica��o de cenas, identifica��o de transi��o de cenas, t�cnica multimodal.